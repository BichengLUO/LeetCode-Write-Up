\section{Math}

\subsection{2. Add Two Numbers}

\paragraph{Description}

You are given two non-empty linked lists representing two non-negative integers. The digits are stored in reverse order and each of their nodes contain a single digit. Add the two numbers and return it as a linked list.

You may assume the two numbers do not contain any leading zero, except the number 0 itself.

\paragraph{Solution}

\underline{Code Hints}:
\begin{itemize}
    \item Use dummy header to generate the result linked list.
    \item Check whether $l_1,l_2,carry$ is null or zero.
    \item Use a pointer pointing the last node to generate the current one and move on.
\end{itemize}

\begin{minted}[linenos]{cpp}
ListNode* addTwoNumbers(ListNode* l1, ListNode* l2) {
    int carry = 0;
    ListNode dummy_head(0); //Use dummy header to generate linked list
    ListNode *p = &dummy_head;
    while (l1 || l2 || carry) //Check whether l1,l2,carry is null or zero
    {
        int result = carry + (l1 ? l1->val : 0) + (l2 ? l2->val : 0);
        carry = result / 10;
        p->next = new ListNode(result % 10);
        
        l1 = l1 ? l1->next : l1;
        l2 = l2 ? l2->next : l2;
        p = p->next;
    }
    return dummy_head.next;
}
\end{minted}

\subsection{13. Roman to Integer}

\paragraph{Description}

Given a roman numeral, convert it to an integer.

Input is guaranteed to be within the range from 1 to 3999.

\paragraph{Solution}
\underline{Code Hints}:
\begin{itemize}
    \item Use initializer list to initialize unordered\_map:\mint{cpp}|unordered_map<int, string> m2 = {{1, "foo"}, {3, "bar"}, {2, "baz"}};|
\end{itemize}

$$\bm{I}=\sum^{n-1}_{i=1}{\bm{S}(m[s[i+1]]<m[s[i]])m[s[i]]}+m[s[n]]$$
$s[i]$ denotes the $i^{th}$ character of the roman numeral string $s$. $m$ is used to map the character to the integer value. The function $\bm{S}(b)$ check the boolean expression $b$, if it is true then return $1$, false return $-1$. 

\subsection{258. Add Digits}

\paragraph{Description}

Given a non-negative integer num, repeatedly add all its digits until the result has only one digit.

For example:

Given num = 38, the process is like: 3 + 8 = 11, 1 + 1 = 2. Since 2 has only one digit, return it.

Follow up:
Could you do it without any loop/recursion in $O(1)$ runtime?

\paragraph{Solution}

Digital Root Congruence Formula:
$$0, 1, 2, 3, 4, 5, 6, 7, 8, 9, 1, 2, 3, 4, 5, 6, 7, 8, 9,\cdots$$
$$dr(n)=1+((n - 1) \% 9)$$

Proof:
$$\overline{abc}\equiv a\cdot 10^2+b\cdot 10+c\equiv a+b+c \pmod{9}$$
Thus, $\overline{abc}$ and $a+b+c$ are in congruence relation. The result of the problem and the input number are in congruence relation, too. The modulus operation with $9$ will lead to the answer. (Be careful to the situation like $9\equiv 0\pmod{9}$)